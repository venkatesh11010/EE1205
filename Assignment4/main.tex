% \iffalse
\documentclass[journal,12pt,twocolumn]{IEEEtran}
\usepackage{cite}
\usepackage{amsmath,amssymb,amsfonts,amsthm}
\usepackage{algorithmic}
\usepackage{graphicx}
\usepackage{textcomp}
\usepackage{xcolor}
\usepackage{txfonts}
\usepackage{listings}
\usepackage{enumitem}
\usepackage{mathtools}
\usepackage{gensymb}
\usepackage{comment}
\usepackage[breaklinks=true]{hyperref}
\usepackage{tkz-euclide}
\usepackage{listings}
\usepackage{gvv}
\def\inputGnumericTable{}
\usepackage[latin1]{inputenc}
\usepackage{color}
\usepackage{array}
\usepackage{longtable}
\usepackage{calc}
\usepackage{multirow}
\usepackage{hhline}
\usepackage{ifthen}
\usepackage{lscape}
\usepackage{caption}

\newtheorem{theorem}{Theorem}[section]
\newtheorem{problem}{Problem}
\newtheorem{proposition}{Proposition}[section]
\newtheorem{lemma}{Lemma}[section]
\newtheorem{corollary}[theorem]{Corollary}
\newtheorem{example}{Example}[section]
\newtheorem{definition}[problem]{Definition}
\newcommand{\BEQA}{\begin{eqnarray}}
\newcommand{\EEQA}{\end{eqnarray}}
\newcommand{\define}{\stackrel{\triangle}{=}}
\theoremstyle{remark}
\newtheorem{rem}{Remark}
\begin{document}

\bibliographystyle{IEEEtran}
\vspace{3cm}

\title{GATE: CE - 30.2023}
\author{EE23BTECH11010 - Venkatesh D Bandawar $^{*}$% <-this % stops a space
}
\maketitle
% \newpage
\bigskip

% \renewcommand{\thefigure}{\theenumi}
% \renewcommand{\thetable}{\theenumi}

\textbf{Question:} In the differential equation $\frac{dy}{dx} + \alpha x y = 0, \alpha$ is a positive constant. If $y = 1.0$ at
$x = 0.0$, and $y = 0.8$ at $x = 1.0$, the value of $\alpha$ is (rounded off to three decimal places).  \hfill(GATE CE 2023)

\solution
\begin{table}[!h] 
\centering
\begin{tabular}{|c|c|c|}
\hline
     \textbf{Corner}&&\\ \textbf{Frequency} & \textbf{Description} & \textbf{Equation} \\
     \hline
     $10$ &  Zero & $\abs{T(s)} = 20.0 (log_{10}(w) - 1.0)$\\
     \hline
     $0.1$ & Pole & $\abs{T(s)} = -20.0 (log_{10}(w) - 0.1)$\\
     \hline
     $0.5$ & Pole & $\abs{T(s)} = -20.0 (log_{10}(w) + log_{10}(0.5))$\\
     \hline
     $100$ & Pole & $\abs{T(s)} = -20.0 (log_{10}(w) - 2.0)$\\
     \hline
\end{tabular}

\caption{Given parameters}
\label{given parameters list.gate.ce.30}
\end{table}

% \begin{align}
%     \frac{dy}{dx} + \alpha x y &= 0\\
%     \int \frac{dy}{y} &= - \int \alpha x dx\\
%     \ln(\abs{y}) &= - \frac{\alpha x^2}{2} + c
% \end{align}
% Substituting $x$ and $y$ values,
% \begin{align}
%     c &= \ln(1) = 0\\
%     \alpha &= -2 \ln(0.8) = 0.446
% \end{align}



\begin{align}
    \frac{dy}{dx} + \alpha x y &= 0
\end{align}
Taking laplace transform,\\
\begin{align}
    sY(s)-y(0) - \alpha \frac{dY(s)}{ds} &=0\\
    \frac{dY(s)}{ds} - \frac{s}{\alpha} Y(s) &= -\frac{Y(0)}{\alpha}\\ 
\end{align}

I.F.= $e^{\int -\frac{s}{\alpha}ds} = e^{-\frac{s^2}{2\alpha}}$
\begin{align}
    e^{-\frac{s^2}{2\alpha}} Y(s) &= \frac{y(0)}{\alpha}\int_{-\infty}^{\infty} e^{-\frac{s^2}{2\alpha}} ds\\
    &= -\frac{y(0)}{\alpha} \sqrt{2\pi\alpha}\\
    Y(s)= -\sqrt{\frac{2\pi}{\alpha}} y(0) e^{\frac{s^2}{2\alpha}}\\
    Y'(s) = -\sqrt{\frac{2\pi}{\alpha}} y(0) e^{\frac{s^2}{2\alpha}} \frac{s}{\alpha}
\end{align}



\end{document}
