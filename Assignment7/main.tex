%\iffalse
\let\negmedspace\undefined
\let\negthickspace\undefined
\documentclass[journal,12pt,twocolumn]{IEEEtran}
\usepackage{cite}
\usepackage{amsmath,amssymb,amsfonts,amsthm}
\usepackage{algorithmic}
\usepackage{graphicx}
\usepackage{textcomp}
\usepackage{xcolor}
\usepackage{txfonts}
\usepackage{listings}
\usepackage{enumitem}
\usepackage{mathtools}
\usepackage{gensymb}
\usepackage{comment}
\usepackage[breaklinks=true]{hyperref}
\usepackage{tkz-euclide} 
\usepackage{listings}
\usepackage{gvv}                            \usepackage{tikz}
\usepackage{circuitikz}
\def\inputGnumericTable{}                                
\usepackage[latin1]{inputenc}                            
\usepackage{color}                                       
\usepackage{array}                                       
\usepackage{longtable}                                   
\usepackage{calc}                              
\usepackage{tikz}
\usepackage{multirow}                                    
\usepackage{hhline}                                      
\usepackage{ifthen}                            
\usepackage{caption}
\usepackage{lscape}
\usepackage{amsmath}
\newtheorem{theorem}{Theorem}[section]
\newtheorem{problem}{Problem}
\newtheorem{proposition}{Proposition}[section]
\newtheorem{lemma}{Lemma}[section]
\newtheorem{corollary}[theorem]{Corollary}
\newtheorem{example}{Example}[section]
\newtheorem{definition}[problem]{Definition}
\newcommand{\BEQA}{\begin{eqnarray}}
\newcommand{\EEQA}{\end{eqnarray}}
\newcommand{\define}{\stackrel{\triangle}{=}}
\theoremstyle{remark}
\newtheorem{rem}{Remark}

\begin{document}

\bibliographystyle{IEEEtran}
\vspace{3cm}

\title{GATE 2022 IN.53}
\author{EE23BTECH11010 - VENKATESH BANDAWAR$^{*}$% <-this % stops a space
}
\maketitle
\newpage
\bigskip
\textbf{Question:} In a unity-gain feedback control system, the plant
$P(s) = \frac{0.001}{s\brak{2s+1}\brak{0.01s+1}}$
is controlled by a lag compensator
$C(s) = \frac{s+10}{s+0.1}$
The slope (in dB/decade) of the asymptotic Bode magnitude plot of the loop gain
at $\omega= 3 $rad/s is \_\_\_\_\_\_\_\_ (in integer)
\hfill(GATE 2022 IN)\\
\solution
\begin{table}[!h]
    \centering
    \begin{tabular}{|c|c|c|}
\hline
     \textbf{Corner}&&\\ \textbf{Frequency} & \textbf{Description} & \textbf{Equation} \\
     \hline
     $10$ &  Zero & $\abs{T(s)} = 20.0 (log_{10}(w) - 1.0)$\\
     \hline
     $0.1$ & Pole & $\abs{T(s)} = -20.0 (log_{10}(w) - 0.1)$\\
     \hline
     $0.5$ & Pole & $\abs{T(s)} = -20.0 (log_{10}(w) + log_{10}(0.5))$\\
     \hline
     $100$ & Pole & $\abs{T(s)} = -20.0 (log_{10}(w) - 2.0)$\\
     \hline
\end{tabular}

    \caption{Given Parameters list}
    \label{tab:Given Parameters list.gate2022.IN.53}
\end{table}

\begin{align}
    Gain(K) = \lim_{s\rightarrow 0} L(s)
\end{align}
Excluding $s$ and $\frac{1}{s}$, From Table \tabref{tab:Given Parameters list.gate2022.IN.53}
\begin{align}
    K &= 0.1
\end{align}
\begin{align}
    \abs{L(s)} &= 20\log_{10} K\\
    &= -20 dB
\end{align}
Here, $10,0,0.5,100,0.1$ are corner frequencies of loop gain L(s) 
% In a Bode magnitude plot,\\
% Zero of L(s) = 10
% \begin{align}
%     \abs{L(s)} &= 20\log_{10} \omega
% \end{align}
% Poles of L(s) = 0, 0.1, 0.5, 100
% \begin{align}
%     \abs{L(s)} &= -20\log_{10} \omega
% \end{align}
\begin{table}[!h]
    \centering
    \begin{tabular}{|c|c|c|}
\hline
     \textbf{Corner}&&\\ \textbf{Frequency} & \textbf{Description} & \textbf{Equation} \\
     \hline
     $10$ &  Zero & $\abs{L(\omega)} = 20.0 * (log_{10}(w) - 1.0)$\\
     \hline
     $0$ & Pole & $\abs{L(\omega)} = -20.0 * log_{10}(w)$\\
     \hline
     $0.1$ & Pole & $\abs{L(\omega)} = -20.0 * (log_{10}(w) - 0.1)$\\
     \hline
     $0.5$ & Pole & $\abs{L(\omega)} = -20.0 * (log_{10}(w) + log_{10}(0.5))$\\
     \hline
     $100$ & Pole & $\abs{L(\omega)} = -20.0 * (log_{10}(w) - 2.0)$\\
     \hline
\end{tabular}

    \caption{Caption}
    \label{tab:corner frequency.gate2022.IN.53}
\end{table}
\begin{table}[!h]
    \centering
    \begin{tabular}{|c|c|c|}
\hline
    \textbf{Parameter} & \textbf{Range} & \textbf{Equation} \\
    \hline
    \multirow{6}{*}{$\abs{L(\omega)}$} & $\omega < 0$ & $0$\\
   \cline{2-3}
   & $0 < \omega < 0.1$ & $-20.0 * log_{10}(w)$ \\
   \cline{2-3}
   & $0.1 < \omega < 0.5 $ & $-20.0 * \brak{2log_{10}(w) - 0.1}$ \\
   \cline{2-3}
   & $0.5 < \omega < 10 $ & $-20.0 * \brak{3log_{10} - 0.1 + log_10{0.5}}$ \\
   \cline{2-3}
   & $10 < \omega < 100 $ & $-20.0 * \brak{2log_{10} + 0.9 + log_10{0.5}}$ \\
   \cline{2-3}
   & $\omega > 100 $ & $-20.0 * \brak{3log_{10} - 1.9 + log_10{0.5}}$ \\
   \hline
    \end{tabular}

    \caption{Caption}
    \label{tab:net bode mag equation.gate2022.IN.53}
\end{table}
\begin{figure}[!h]
    \centering
    \includegraphics[width=\columnwidth]{figs/bode_mag_plot.png}
    \caption{Pink Line = Bode magnitude plot of loop gain}
\end{figure}

Slope of Bode magnitude plot (at $\omega=3$) = $-60$ dB/decade
\end{document}
