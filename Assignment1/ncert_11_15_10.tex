% \iffalse
\let\negmedspace\undefined
\let\negthickspace\undefined
\documentclass[journal,12pt,twocolumn]{IEEEtran}
\usepackage{cite}
\usepackage{amsmath,amssymb,amsfonts,amsthm}
\usepackage{algorithmic}
\usepackage{graphicx}
\usepackage{textcomp}
\usepackage{xcolor}
\usepackage{txfonts}
\usepackage{listings}
\usepackage{enumitem}
\usepackage{mathtools}
\usepackage{gensymb}
\usepackage{comment}
\usepackage[breaklinks=true]{hyperref}
\usepackage{tkz-euclide} 
\usepackage{listings}
\usepackage{gvv}                                        
\def\inputGnumericTable{}                                 
\usepackage[latin1]{inputenc}                                
\usepackage{color}                                            
\usepackage{array}                                            
\usepackage{longtable}                                       
\usepackage{calc}                                             
\usepackage{multirow}                                         
\usepackage{hhline}                                           
\usepackage{ifthen}                                           
\usepackage{lscape}

\newtheorem{theorem}{Theorem}[section]
\newtheorem{problem}{Problem}
\newtheorem{proposition}{Proposition}[section]
\newtheorem{lemma}{Lemma}[section]
\newtheorem{corollary}[theorem]{Corollary}
\newtheorem{example}{Example}[section]
\newtheorem{definition}[problem]{Definition}
\newcommand{\BEQA}{\begin{eqnarray}}
\newcommand{\EEQA}{\end{eqnarray}}
\newcommand{\define}{\stackrel{\triangle}{=}}
\theoremstyle{remark}
\newtheorem{rem}{Remark}
\begin{document}

\bibliographystyle{IEEEtran}
\vspace{3cm}

\title{NCERT 11.15. Q10}
\author{EE23BTECH11010 - Venkatesh Bandawar$^{*}$% <-this % stops a space
}
\maketitle
\newpage
\bigskip

\renewcommand{\thefigure}{\theenumi}
\renewcommand{\thetable}{\theenumi}

\bibliographystyle{IEEEtran}

\parindent 0px
\textbf{Question:} For the travelling harmonic wave
$y\brak {x, t} = 2.0 cos 2\pi \brak{10t - 0.0080 x + 0.35}$ where x and y are in cm and t in s. Calculate the phase difference between oscillatory
motion of two points separated by a distance of 

\begin{enumerate} [label=(\alph*)]
    \item 4 m
    \item 0.5 m
    \item $\lambda$/2
    \item 3$\lambda$/4
\end{enumerate}

\textbf{Solution:}  
\begin{table}[htbp] \small
\centering
\begin{tabular}{|c|c|c|}
\hline
\textbf{symbol}&\textbf{parameter} &\textbf{value}\\
   \hline
   k & angular wave number & 2$\pi$\brak{0.008}\\
   \hline
   $\lambda$  & wavelength & 125cm\\
   \hline
   $\omega$ & angular frequency & 2$\pi$\brak{10}\\
   \hline
   A & amplitude & 2.0\\
   \hline
   $\phi$ & phase &  2$\pi$\brak{0.35} \\
   \hline
\end{tabular}

\end{table}
Harmonic wave :
\begin{equation}
    y\brak{x, t} = 2.0 \cos 2\pi \brak{10t - 0.0080 x + 0.35}
\end{equation}

Phase of harmonic wave (at x):
\begin{equation}
    = 2\pi \brak{10t - 0.0080x + 0.35}
\end{equation}
\begin{align}
    k &= 2\pi(0.008) \label{} \\
    \because k &= \dfrac{2\pi}{\lambda} \\
    \lambda &= \dfrac{2\pi}{k} \\
    \lambda &= \dfrac{2\pi}{2 \pi \times 0.008}\nonumber \\
    \lambda &= 125 \text{cm}
\end{align}
\begin{align}
    \text{Phase difference} &= \theta_1 - \theta_2 \\
    &= \brak{kx_1 + \omega t + \phi} - \brak{kx_2 + \omega t + \phi}\\
    &= k\brak{x_1 - x_2} \label{equation for phase difference}
\end{align}

\begin{enumerate} [label=(\alph*)]
   \item
$\because \brak{x_1 - x_2} = 400\, cm$ 
\begin{align}
    \text{phase difference} &= k\brak{x_1 - x_2} \tag{\ref{equation for phase difference}} \nonumber\\
    &= 2\pi \times 0.0080 \times 400 \nonumber\\
    &= 6.4 \pi\, \text{radians}
\end{align}

   \item 
$\because \brak{x_1 - x_2} = 50\, cm$ 
\begin{align}
    \text{phase difference} &= k\brak{x_1 - x_2} \tag{\ref{equation for phase difference}} \nonumber\\
    &= 2\pi \times 0.0080 \times 50 \nonumber\\
    &= 0.8\pi \, \text{radians}
\end{align}

    \item $\because \brak{x_1 - x_2} = \dfrac{\lambda}{2}\, cm$ 
\begin{align}
    \text{phase difference} &= k\brak{x_1 - x_2} \tag{\ref{equation for phase difference}} \nonumber\\
    &= 2\pi \times 0.0080 \times \dfrac{\lambda}{2} \\
    &= 2\pi \times 0.0080 \times \dfrac{125}{2} 
    \brak{\because \lambda=125} \nonumber \\
    &= \pi \text{radians}
\end{align}
    
    \item $\because \brak{x_1 - x_2} = \dfrac{3\lambda}{4}\, cm$ 
\begin{align}
    \text{phase difference} &= k\brak{x_1 - x_2} \tag{\ref{equation for phase difference}} \nonumber\\
    &= 2\pi \times 0.0080 \times 3 \times \dfrac{\lambda}{4} \\
    &= 2\pi \times 0.0080 \times 3 \times \dfrac{125}{4}  \brak{\because\lambda=125} \nonumber\\
    &= \dfrac{3\pi}{2} \text{radians}
\end{align}

\end{enumerate}

\end{document}
