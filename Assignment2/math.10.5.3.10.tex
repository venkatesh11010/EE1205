% \iffalse
\let\negmedspace\undefined
\let\negthickspace\undefined
\documentclass[journal,12pt,twocolumn]{IEEEtran}
\usepackage{cite}
\usepackage{amsmath,amssymb,amsfonts,amsthm}
\usepackage{algorithmic}
\usepackage{graphicx}
\usepackage{textcomp}
\usepackage{xcolor}
\usepackage{txfonts}
\usepackage{listings}
\usepackage{enumitem}
\usepackage{mathtools}
\usepackage{gensymb}
\usepackage{comment}
\usepackage[breaklinks=true]{hyperref}
\usepackage{tkz-euclide} 
\usepackage{listings}
\usepackage{gvv}                                        
\def\inputGnumericTable{}                                 
\usepackage[latin1]{inputenc}                                
\usepackage{color}                                            
\usepackage{array}                                            
\usepackage{longtable}                                       
\usepackage{calc}                                             
\usepackage{multirow}                                         
\usepackage{hhline}                                           
\usepackage{ifthen}                                           
\usepackage{lscape}
\usepackage{amsmath}
\usepackage{caption}
\newtheorem{theorem}{Theorem}[section]
\newtheorem{problem}{Problem}
\newtheorem{proposition}{Proposition}[section]
\newtheorem{lemma}{Lemma}[section]
\newtheorem{corollary}[theorem]{Corollary}
\newtheorem{example}{Example}[section]
\newtheorem{definition}[problem]{Definition}
\newcommand{\BEQA}{\begin{eqnarray}}
\newcommand{\EEQA}{\end{eqnarray}}
\newcommand{\define}{\stackrel{\triangle}{=}}
\theoremstyle{remark}
\newtheorem{rem}{Remark}
\begin{document}

\bibliographystyle{IEEEtran}
\vspace{3cm}

\title{NCERT 10.5.3 10Q}
\author{EE22BTECH11010 - Venkatesh D Bandawar $^{*}$% <-this % stops a space
}
\maketitle
\newpage
\bigskip

\renewcommand{\thefigure}{\theenumi}
\renewcommand{\thetable}{\theenumi}

\parindent 0px
\textbf{Question:} Show that $a_1$
, $a_2$
, . . ., $a_n$
, . . . form an AP where an is defined as below :\\
(i) $a_n$ = $3 + 4n$ \\ (ii) $a_n$ = $9 - 5n$ \\
Also find the sum of the first 15 terms in each case.

\vspace{0.5cm}
\textbf{Answer:} 
\begin{enumerate} [label=(\roman*)]
    \item Given: $\,$ $a_n$ = $3 + 4n$ \\
    we know that, The AP has constant common difference between two consecutive terms
    \begin{align}
        \therefore \text{common difference } (d) &= a_{n+1} - a_n \\
        &= \brak{3+4(n+1)} \nonumber\\
        & \,\,\, -\brak{3+4n} \nonumber\\
        &= 4
    \end{align}
   $\because$ Given equation has common difference between any two consecutive terms is $4$ i.e. independent of '$n$'
   \begin{center}
       Hence given sequence is in AP.
   \end{center}
   \vspace{0.3cm}

    \begin{table}[htbp] 
    \centering
    \begin{tabular}{|c|c|c|}
\hline
     \textbf{Corner}&&\\ \textbf{Frequency} & \textbf{Description} & \textbf{Equation} \\
     \hline
     $10$ &  Zero & $\abs{T(s)} = 20.0 (log_{10}(w) - 1.0)$\\
     \hline
     $0.1$ & Pole & $\abs{T(s)} = -20.0 (log_{10}(w) - 0.1)$\\
     \hline
     $0.5$ & Pole & $\abs{T(s)} = -20.0 (log_{10}(w) + log_{10}(0.5))$\\
     \hline
     $100$ & Pole & $\abs{T(s)} = -20.0 (log_{10}(w) - 2.0)$\\
     \hline
\end{tabular}

    \vspace{0.2cm}
    \caption{\normalsize\textsl{Given \, parameters in $1^{st}$ AP}}
    \label{given parameters list}
    \end{table}

    Z - Transformation of $a_n$ :
    \begin{align}
        \because X(z) &= \sum_{n = -\infty}^{\infty}x(n) \, z^{-n} \\
         x(n)&= a_n\,u(n) \\
        & = \sbrak{3+4n} \, u(n)
    \end{align}

    \[
    \text{Here},\, u(n) =
    \begin{cases}
        0 & \text{for } n < 0 \\
        1 & \text{for } n \geq 0 
    \end{cases}
    \]

    \begin{figure}[!h] 
    \centering
    \includegraphics[width=\columnwidth]{figs/signal_x1.png}
    \caption{}
    \label{fig:Graph1}
    \end{figure}
    \begin{align}
        \implies  X(z) &= \sum_{n = -\infty}^{\infty}[3 + 4n].u(n) \, z^{-n} \\
        \implies X(z) &= 3 \sum_{n = 0}^{\infty} 1.z^{-n} + 4 \sum_{n = 0}^{\infty} 1.n.z^{-n}
    \end{align}
    \begin{align}
        \text{Since,} \nonumber\\
        &\sum_{n=0}^{\infty} r^n = \dfrac{1}{1-r}  \,\,\, 
        ; |r|<1 \\
        &\text{Differentiating w.r.t. 'r' and multiplying by 'r' }\nonumber \\
        &\text{we get,} \nonumber \\
        &\sum_{n=0}^{\infty} n.r^n = \dfrac{r}{(1-r)^2}
    \end{align}

    \begin{align}
        &\text{Here}, r = z^{-1} \nonumber\\
        &\implies X(z) = \dfrac{3}{1-z^{-1}} + \dfrac{4.z^{-1}}{(1-z^{-1})^2}\\
        &\{ Where, |z|>1\} \nonumber
    \end{align}
    
    \item Given: $\,$ $a_n$ = $9 - 5n$ \\
    we know that, The AP has constant common difference between two consecutive terms
    \begin{align}
        \therefore \text{common difference } (d) &= a_{n+1} - a_n \\
        &= \brak{9-5(n+1)} - \brak{9-5n} \nonumber\\
        &= -5
    \end{align}
   $\because$ Given equation has common difference between any two consecutive terms is $-5$ i.e. independent of '$n$'
   \begin{center}
       Hence given sequence is in AP.
   \end{center}

    \begin{table}[htbp] 
    \centering
    \begin{tabular}{|c|c|c|}
\hline
     \textbf{Corner}&&\\ \textbf{Frequency} & \textbf{Description} & \textbf{Equation} \\
     \hline
     $10$ &  Zero & $\abs{L(\omega)} = 20.0 * (log_{10}(w) - 1.0)$\\
     \hline
     $0$ & Pole & $\abs{L(\omega)} = -20.0 * log_{10}(w)$\\
     \hline
     $0.1$ & Pole & $\abs{L(\omega)} = -20.0 * (log_{10}(w) - 0.1)$\\
     \hline
     $0.5$ & Pole & $\abs{L(\omega)} = -20.0 * (log_{10}(w) + log_{10}(0.5))$\\
     \hline
     $100$ & Pole & $\abs{L(\omega)} = -20.0 * (log_{10}(w) - 2.0)$\\
     \hline
\end{tabular}

    \vspace{0.2cm}
    \caption{\normalsize\textsl{Given \, parameters in $2^{st}$ AP}}
    \label{given parameters list}
    \end{table}

    Z - Transformation of $a_n$ :
    \begin{align}
        \because X(z) &= \sum_{n = -\infty}^{\infty}x(n) \, z^{-n} \\
         x(n)&= a_n\,u(n) \\
        & = \sbrak{9-5n} \, u(n)
    \end{align}

    \[
    \text{Here},\, u(n) =
    \begin{cases}
        0 & \text{for } n < 0 \\
        1 & \text{for } n \geq 0 
    \end{cases}
    \]

    \begin{figure}[!h] 
    \centering
    \includegraphics[width=\columnwidth]{figs/signal_x2.png}
    \caption{}
    \label{fig:Graph2}
    \end{figure}
    \begin{align}
        \implies  X(Z) &= \sum_{n = -\infty}^{\infty}[9 - 5n].u(n) \, z^{-n} \\
        \implies X(z) &= 9 \sum_{n = 0}^{\infty} 1.z^{-n} - 5 \sum_{n = 0}^{\infty} 1.n.z^{-n}
    \end{align}
    \begin{align}
        \text{Since,} \nonumber\\
        &\sum_{n=0}^{\infty} r^n = \dfrac{1}{1-r}  \,\,\, 
        ; |r|<1 \\
        &\text{Differentiating w.r.t. 'r' and multiplying by 'r' }\nonumber \\
        &\text{we get,} \nonumber \\
        &\sum_{n=0}^{\infty} n.r^n = \dfrac{r}{(1-r)^2}
    \end{align}

    \begin{align}
        &\text{Here}, r = z^{-1} \nonumber\\
        &\implies X(z) = \dfrac{9}{1-z^{-1}} - \dfrac{5.z^{-1}}{(1-z^{-1})^2}\\
        &\{ Where, |z|>1\} \nonumber
    \end{align}

\end{enumerate}

\end{document}
