%\iffalse
\let\negmedspace\undefined
\let\negthickspace\undefined
\documentclass[journal,12pt,twocolumn]{IEEEtran}
\usepackage{cite}
\usepackage{amsmath,amssymb,amsfonts,amsthm}
\usepackage{algorithmic}
\usepackage{graphicx}
\usepackage{textcomp}
\usepackage{xcolor}
\usepackage{txfonts}
\usepackage{listings}
\usepackage{enumitem}
\usepackage{mathtools}
\usepackage{gensymb}
\usepackage{comment}
\usepackage[breaklinks=true]{hyperref}
\usepackage{tkz-euclide} 
\usepackage{listings}
\usepackage{gvv}                            \usepackage{tikz}
\usepackage{circuitikz}
\def\inputGnumericTable{}                                
\usepackage[latin1]{inputenc}                            
\usepackage{color}                                       
\usepackage{array}                                       
\usepackage{longtable}                                   
\usepackage{calc}                              
\usepackage{tikz}
\usepackage{multirow}                                    
\usepackage{hhline}                                      
\usepackage{ifthen}                            
\usepackage{caption}
\usepackage{lscape}
\usepackage{amsmath}
\newtheorem{theorem}{Theorem}[section]
\newtheorem{problem}{Problem}
\newtheorem{proposition}{Proposition}[section]
\newtheorem{lemma}{Lemma}[section]
\newtheorem{corollary}[theorem]{Corollary}
\newtheorem{example}{Example}[section]
\newtheorem{definition}[problem]{Definition}
\newcommand{\BEQA}{\begin{eqnarray}}
\newcommand{\EEQA}{\end{eqnarray}}
\newcommand{\define}{\stackrel{\triangle}{=}}
\theoremstyle{remark}
\newtheorem{rem}{Remark}

\begin{document}

\bibliographystyle{IEEEtran}
\vspace{3cm}

\title{GATE 2022 BM.38}
\author{EE23BTECH11010 - VENKATESH BANDAWAR$^{*}$% <-this % stops a space
}
\maketitle
\newpage
\bigskip
\textbf{Question:} An input $x(t)$ is applied to a system with a frequency transfer function given by
$H(j\omega)$ as shown below. The magnitude and phase response of the transfer function
are shown below. If $y(t_d) = 0$ for $x(t) = u(t)$, the time $t_d (> 0)$ is.\\ 
\hfill(Gate 2022 BM.38)
\begin{figure}[!h] 
\centering
\includegraphics[width=\columnwidth]{figs/given_graph.png}
\caption{Graph of y(t)}
\label{fig:Graph1_gate_CE_30}
\end{figure}

\solution 
\begin{table}[!h]
    \centering
    \begin{tabular}{|c|c|c|}
\hline
     \textbf{Corner}&&\\ \textbf{Frequency} & \textbf{Description} & \textbf{Equation} \\
     \hline
     $10$ &  Zero & $\abs{T(s)} = 20.0 (log_{10}(w) - 1.0)$\\
     \hline
     $0.1$ & Pole & $\abs{T(s)} = -20.0 (log_{10}(w) - 0.1)$\\
     \hline
     $0.5$ & Pole & $\abs{T(s)} = -20.0 (log_{10}(w) + log_{10}(0.5))$\\
     \hline
     $100$ & Pole & $\abs{T(s)} = -20.0 (log_{10}(w) - 2.0)$\\
     \hline
\end{tabular}

    \caption{Input Parameters Table}
\end{table}
\begin{align}
    \because \abs{H\brak{f}} &= 1\\
    \abs{z} &= \abs{\bar{z}}\\
    \therefore H(f) &= \frac{a - j2\pi f}{a + j2\pi f}\\
    \angle {H(f)} &= \tan^{-1}\brak{\frac{-2\pi f}{a}} - \tan^{-1}\brak{\frac{2\pi f}{a}}\\
    &= -2 \tan^{-1}\brak{\frac{2\pi f}{a}}
\end{align}
At $2\pi f = 10^4, \angle{H(f)} = -\frac{\pi}{2}$
\begin{align}
    \implies a = 10^4 
\end{align}
\begin{figure}[!h]
    \centering
    \includegraphics[width=\columnwidth]{figs/phase.png}
    \caption{Phase of H(f)}
\end{figure}
\begin{align}
    u(t) &\system{\mathcal{L}} \frac{1}{j2\pi f}\\
    Y(f) &= \frac{1}{j2\pi f} \frac{a - j2\pi f}{a + j2\pi f}\\
    &= \frac{1}{j2\pi f} - \frac{2}{a + j2\pi f}\\
    \frac{1}{j2\pi f} &\mathrel{\substack{\mathcal{F}^{-1}\\\longleftrightarrow}} u(t)\\
    \frac{1}{a + j2\pi f} &\mathrel{\substack{\mathcal{F}^{-1}\\\longleftrightarrow}} e^{-at} u(t) \\
    y(t) &= (1-2e^{-at}) u(t)
\end{align}
\begin{align}
    \because y(t_d) &= 0 \\
    t_d &= 100 \ln{2} \micro s 
\end{align}
\begin{figure}[!h]
    \centering
    \includegraphics[width=\columnwidth]{figs/graph.png}
\end{figure}
\end{document}
