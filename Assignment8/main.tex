%\iffalse
\let\negmedspace\undefined
\let\negthickspace\undefined
\documentclass[journal,12pt,twocolumn]{IEEEtran}
\usepackage{cite}
\usepackage{amsmath,amssymb,amsfonts,amsthm}
\usepackage{algorithmic}
\usepackage{graphicx}
\usepackage{textcomp}
\usepackage{xcolor}
\usepackage{txfonts}
\usepackage{listings}
\usepackage{enumitem}
\usepackage{mathtools}
\usepackage{gensymb}
\usepackage{comment}
\usepackage[breaklinks=true]{hyperref}
\usepackage{tkz-euclide} 
\usepackage{listings}
\usepackage{gvv}                            \usepackage{tikz}
\usepackage{circuitikz}
\def\inputGnumericTable{}                                
\usepackage[latin1]{inputenc}                            
\usepackage{color}                                       
\usepackage{array}                                       
\usepackage{longtable}                                   
\usepackage{calc}                              
\usepackage{tikz}
\usepackage{multirow}                                    
\usepackage{hhline}                                      
\usepackage{ifthen}                            
\usepackage{caption}
\usepackage{lscape}
\usepackage{amsmath}
\newtheorem{theorem}{Theorem}[section]
\newtheorem{problem}{Problem}
\newtheorem{proposition}{Proposition}[section]
\newtheorem{lemma}{Lemma}[section]
\newtheorem{corollary}[theorem]{Corollary}
\newtheorem{example}{Example}[section]
\newtheorem{definition}[problem]{Definition}
\newcommand{\BEQA}{\begin{eqnarray}}
\newcommand{\EEQA}{\end{eqnarray}}
\newcommand{\define}{\stackrel{\triangle}{=}}
\theoremstyle{remark}
\newtheorem{rem}{Remark}

\begin{document}

\bibliographystyle{IEEEtran}
\vspace{3cm}

\title{GATE 2021 BM.29}
\author{EE23BTECH11010 - VENKATESH BANDAWAR$^{*}$% <-this % stops a space
}
\maketitle
\newpage
\bigskip
\textbf{Question:} In the circuit shown below, $V_s =100V, R_1 =30\Omega, R_2 =60\Omega, R_3 =90\Omega, R_4 =45\Omega, R_5 =30\Omega$ The current flowing through resistor $R_3$ is \_\_\_\_\_\_A.  \hfill(Gate 2022 BM) \\
\begin{figure}[!ht]
    \centering
    \begin{circuitikz}[scale = 0.8]
    % Nodes
    \draw (0,0) to[battery] (0,-6);
     \node at (-1,-3) {$V_s$};
    \draw (0,0) -- (4,0);
    \draw (4,0) -- (7,0);
    \draw (0,-6) -- (4,-6);
    \draw (4,-6) -- (7,-6);
    
    % Resistors
    \draw (4,0) to[R, l=$R_1$] (4,-3);
    \draw (7,0) to[R, l=$R_2$] (7,-3);
    \draw (4,-3) to[R, l=$R_3$] (7,-3);
    \draw [->] (6.8,-3) -- (7,-3);
    \draw (4,-3) to[R, l=$R_4$] (4,-6);
    \draw (7,-3) to[R, l=$R_5$] (7,-6);

    \draw [dotted,line width = 1.5pt] (1.5,-7) -- (1.5,-5);
    \draw [dotted,line width = 1.5pt,->] (1.5,-5) -- (2.2,-5);
    \node at (2.2,-7) {$R_{e}$};
\end{circuitikz}

    \caption{Circuit Digram}
    \label{given diagram_2022_bm_29}
\end{figure}

\solution
\begin{table}[!h]
    \centering
    \begin{tabular}{|c|c|c|}
\hline
    Parameter & Description & Value\\
    \hline
    $P(s)$ & Plant Transfer Function & $\frac{0.001}{s\brak{\frac{s}{0.5}+1}\brak{\frac{s}{100}+1}}$\\
    \hline
    $C(s)$ & Lag Compensator  & $\frac{100\brak{\frac{s}{10}+1}}{\frac{s}{0.1}+1}$\\
    \hline
    $T(s)$ & Loop gain  & $P(s) C(s)$ \\
    \hline
    $\omega$ & Angular Frequency & 3rad/s \\
    \hline
\end{tabular}

    \caption{Given Parameters}
    \label{Given Parameters_2022_bm_29}
\end{table}
\begin{figure}[!h]
    \centering
    \begin{circuitikz}[scale = 0.8]
    \draw (4,-3) to[R, l=$R_{4}$] (4,-6);
    \draw (8,-3) to[R, l=$R_5$] (8,-6);
     \draw (0,0) to[battery] (0,-6);
     \node at (-1,-3) {$V_s$};

     \draw (0,0) -- (6,0);
       \draw(6,0) to [R](6,-2);
       \draw (6,-2) to [R](4,-3);
       \draw (6,-2) to [R] (8,-3);
       % \draw (4,-3) to [R](8,-3);
       \draw (0,-6) -- (8,-6);

       \node at (6.8,-1) {$R_{s_{1}}$};
       \node at (4.8,-1.8) {$R_{s_{2}}$};
       \node at (7.2,-2) {$R_{s_{3}}$};
\end{circuitikz}

    \caption{circuit diagram}
    \label{star_delta_2022_bm_29}
\end{figure}
\begin{align}
    R_{s_1} &= \frac{R_1 R_2}{R_1 + R_2 + R_3}\\
    &= 10\Omega\\
    R_{s_2} &= \frac{R_3 R_1}{R_1 + R_2 + R_3}\\
    &= 15\Omega\\
    R_{s_3} &= \frac{R_2 R_3}{R_1 + R_2 + R_3}\\
    &= 30\Omega\\
    R_{e} &= R_{s_1} + \frac{\brak{R_{s_2} + R_4} \brak{R_{s_3} + R_5}}{\brak{R_{s_2} + R_4}+\brak{R_{s_3} + R_5}}\\
    &= 40\Omega\\
    I &= \frac{V_s}{R_e}\\
    &= 2.5A
\end{align}
From Circuit \figref{star_delta_2022_bm_29}
\begin{align}
    V_{R_4} &= \frac{\brak{R_{s_3} + R_5}}{\brak{R_{s_2} + R_4}+\brak{R_{s_3} + R_5}} I R_4\\
    &= 56.25 V
\end{align}
By KVL, 
\begin{align}
    V_{R_1} &= V - V_{R_4}\\
    &= 43.75 V
\end{align}
By KCL,
\begin{align}
    I_{R_3} &= \frac{V_{R_1}}{R_1} - \frac{V_{R_4}}{R_4}\\
    &= 0.2083 A
\end{align}


\end{document}

